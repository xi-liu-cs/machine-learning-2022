\documentclass[12pt, border = 4pt, multi]{article} % \documentclass[tikz, border = 4pt, multi]{article}
\usepackage[a4paper, margin = 70pt]{geometry}
\usepackage{lingmacros}
\usepackage{tree-dvips}
\usepackage{amssymb} % mathbb{}
\usepackage[dvipsnames]{xcolor}
\usepackage{forest}
\usepackage[pdftex]{hyperref}
\usepackage{amsmath} % matrices
\usepackage{xeCJK}
\usepackage{tikz}
\usepackage[arrowdel]{physics}
\usepackage{graphicx}
\usepackage{wrapfig}
\usepackage{listings}
\usepackage{pgfplots, pgfplotstable}
\usepackage{diagbox} % diagonal line in cell
\usepackage[usestackEOL]{stackengine}
\usepackage{makecell}
\usepackage{multirow}
\usepackage{multicol}
\usepackage[T1]{fontenc} 
\setlength{\columnsep}{1cm}
\graphicspath{{./img}} % specify the graphics path to be relative to the main .tex file, denoting the main .tex file directory as ./
\definecolor{orchid}{rgb}{0.7, 0.4, 1.1}
\lstset
{ 
  backgroundcolor = \color{white},
  basicstyle = \scriptsize,
  breaklines = true,
  commentstyle = \color{comment_color}\textit,
  keywordstyle = \color{keyword_color}\bfseries,
  language = c++,
  escapeinside = {\%*}{*)},          
  extendedchars = true,              
  frame = tb,
  numberstyle = \tiny\color{comment_color},
  rulecolor = \color{black},
  showstringspaces = false,
  stringstyle = \color{string_color},
  upquote = true, 
}
\usepackage{xcolor}
\definecolor{comment_color}{rgb}{0, 0.5, 0}
\definecolor{keyword_color}{rgb}{0.3, 0, 0.6}
\definecolor{string_color}{rgb}{0.5, 0, 0.1}
\begin{document}
\section*{Xi Liu}
\subsubsection*{Question 1}
Two players take turns trying to 
kick a ball into the net in soccer. 
Player 1 succeeds with probability 
1/5 and Player 2 succeeds with 
the probability 1/4. Whoever 
succeeds first wins the game 
and the game is over. 
Assuming that Player 1 takes the 
first shot, what is the 
probability that Player 1 wins 
the game? Please derive your 
answer.\\
\\
$p_1$ be probability of success of player 1, $p_1 = 1 / 5$\\
$p_2$ be probability of success of player 2, $p_2 = 1/ 4$\\
\begin{align*}
\sum_{i = 1} ^ {\infty} p_1 (1 - p_1) ^ {i - 1} (1 - p_2) ^ {i - 1}
&= \sum_{i = 1} ^ {\infty} \left(\frac{1}{5}\right) \left(1 - \frac{1}{5}\right) ^ {i - 1} \left(1 - \frac{1}{4}\right) ^ {i - 1}\\
&= \sum_{i = 1} ^ {\infty} \left(\frac{1}{5}\right) \left(\frac{4}{5}\right) ^ {i - 1} \left(\frac{3}{4}\right) ^ {i - 1}\\
&= \frac{1}{5} \sum_{i = 1} ^ {\infty} \left(\frac{4}{5}\right) ^ {i - 1} \left(\frac{3}{4}\right) ^ {i - 1}\\
&= \frac{1}{5} \sum_{i = 1} ^ {\infty}\left(\frac{4}{5}\cdot\frac{3}{4}\right) ^ {i - 1}\\
&= \frac{1}{5} \sum_{i = 1} ^ {\infty}\left(\frac{3}{5}\right) ^ {i - 1}\\
&= \frac{1}{5} \sum_{i = 0} ^ {\infty}\left(\frac{3}{5}\right) ^ i\\
&\text{/* geometric series sum} \sum_{i = 0} ^ n ar ^ i = a\left(\frac{1 - r ^ {n + 1}}{1 - r}\right)\\
&\text{if } |r| < 1 \qquad \lim_{n \rightarrow \infty} \sum_{i = 0} ^ n ar ^ i = \frac{a}{1 - r}\text{ */}\\
&= \frac{1}{5} \cdot \frac{1}{1 - 3 / 5}\\
&= \frac{1}{5} \cdot \frac{1}{2 / 5}\\
&= \frac{1}{5} \cdot \frac{5}{2}\\
&= \boxed{\frac{1}{2}}
\end{align*}
\newpage
\noindent
\subsubsection*{Question 2}
You know that 1\% of the 
population have COVID. You also 
know that 90\% of the people who 
have COVID get a positive test result
and 10\% of people who do not 
have COVID also test positive.
What is the probability that you 
have COVID given that you 
tested positive?\\
\\
let $A$ denote the event that you tested positive\\
$B$ be the event that you have covid\\
$P(B) = 0.01\\
P(B ^ c) = 1 - P(B) = 1 - 0.01 = 0.99\\
P(A | B) = 0.9\\
P(A | B ^ c) = 0.1$
\begin{align*}
P(A | B) &= \frac{P(A \cap B)}{P(B)}, \qquad P(A | B ^ c) = \frac{P(A \cap B ^ c)}{P(B ^ c)}\\
P(A \cap B) &= P(A | B)P(B), \qquad P(A \cap B ^ c) = P(A | B ^ c)P(B ^ c)\\
P(A) &= P(A \cap B) + P(A \cap B ^ c) = P(A | B)P(B) + P(A | B ^ c)P(B ^ c)\\
P(B | A) &= \frac{P(A \cap B)}{P(A)}\\
&= \frac{P(A | B)P(B)}{P(A | B)P(B) + P(A | B ^ c)P(B ^ c)}\\
&= \frac{(0.9)(0.01)}{(0.9)(0.01) + (0.1)(0.99)}\\
&= \frac{0.009}{0.108}\\
&= \frac{9}{108}\\
&= \boxed{\frac{1}{12}}
\end{align*}
\newpage
\noindent
\subsubsection*{Question 3}
Let the function $f(x)$ be defined as: 
\begin{equation}
    f(x) = 
    \left\{ \begin{array}{cl}
    0 & for \,\, x < 0 \\ 
\frac{1}{(1+x)} & otherwise.
\end{array} \right.
\end{equation}
Is $f(x)$ a PDF? If yes, then 
prove that it is a PDF. 
If no, then 
prove that it is not a PDF.\\
\\
let $X$ be a continuous random variable, if $func$ is a probability density function, then $func$ has the property
\begin{align*}
P(-\infty \leq X \leq \infty) &= \int_{-\infty} ^ {\infty} func(x) dx = 1\\
\\
I := \int_{-\infty} ^ {\infty} f(x) dx &= \int_0 ^ {\infty} \frac{1}{1 + x} dx\\
\\
u := 1 + x, \quad du = dx\\
\int \frac{1}{1 + x} dx &= \int \frac{du}{u}\\
&= \ln(u)\\
&= \ln(1 + x)\\
\\
I &= \lim_{a \rightarrow \infty}[\ln(1 + x)]_0 ^ {a}\\
&= \lim_{a \rightarrow \infty}\ln(a) - \ln(1)\\
&= \lim_{a \rightarrow \infty}\ln(a) - 0\\
&= \lim_{a \rightarrow \infty}\ln(a)\\
&= \infty
\end{align*}
since the integral $I$ is not equal to 1, $f(x)$ is not a probability density function\\
\newpage
\noindent
\subsubsection*{Question 4 (10 points)}
Assume that $X$ and $Y$ are two 
independent random variables and 
both have the same density function: 
\begin{equation}
    f(x) =
    \left\{ \begin{array}{cll}
    2x & if & 0 \le x \le 1 \\ 
    0  & otherwise & 
\end{array} \right.
\end{equation}
What is the value of 
$\mathbb{P}(X + Y \le 1)$?
\\
\begin{align*}
P(X + Y \le 1) &= \int_0 ^ 1 \int_0 ^ {1 - y} f_{X, Y}(x, y) dx dy\\
&\text{/* since $X, Y$ are independent, $f_{X, Y}(x, y) = f_X(x)f_Y(y)$ */}\\
&= \int_0 ^ 1 \int_0 ^ {1 - y} f_X(x)f_Y(y) dx dy\\
&= \int_0 ^ 1 \int_0 ^ {1 - y} (2x)(2y) dx dy\\
&= \int_0 ^ 1 \int_0 ^ {1 - y} 4xy\; dx dy\\
&= \int_0 ^ 1 [2x ^ 2 y]_{x = 0} ^ {x = 1 - y} dy\\
&= 2\int_0 ^ 1 (1 - y) ^ 2 y\;dy\\
&= 2\int_0 ^ 1 (y - 2y ^ 2 + y ^ 3) dy\\
&= 2\left[\frac{y ^ 2}{2} - \frac{2}{3}y ^ 3 + \frac{y ^ 4}{4}\right]_0 ^ 1\\
&= 2\left(\frac{1}{2} - \frac{2}{3} + \frac{1}{4}\right)\\
&= \frac{2}{12}\\
&= \boxed{\frac{1}{6}}
\end{align*}
\newpage
\noindent
\subsubsection*{Question 5}
Let $X$ be a random variable which belongs 
to a Uniform distribution between 
$0$ and $1$: $X \sim Unif(0,1)$. Let 
$Y = g(X) = e^{X}$. What is the value 
of $\mathbb{E}(Y)$?\\
\\
$a, b \in \mathbb{R}$, for $Unif(a, b)$
\[f(x) = \frac{1}{b - a}\]
for $Unif(0, 1)$
\[f(x) = \frac{1}{1 - 0} = 1\]
\begin{align*}
E[X] &= \int_{-\infty} ^ {\infty} x f(x) dx\\
E[Y] &= E[g(X)] = \int_{-\infty} ^ {\infty} g(x) f(x) dx\\
&= \int_{-\infty} ^ {\infty} e ^ x (1) dx\\
&= \int_0 ^ 1 e ^ x dx\\
&= [e ^ x]_0 ^ 1\\
&= e ^ 1 - e ^ 0\\
&= \boxed{e - 1}\\
\end{align*}
\newpage
\noindent
\subsubsection*{Question 6}
Suppose that the number of errors per 
computer program has a Poisson 
distribution with mean 5. 
We have 125 program submissions. 
Let $X_1, X_2, \ldots, X_{125}$ 
denote the number of errors in the 
programs. What is the value of 
$\mathbb{P}(\bar{X}_n < 5.5)$?\\
\\
central limit theorem: let $X_1, X_2, ..., X_n$ be random variables, each $X_i$ has expected value $\mu$, variance $\sigma ^ 2$ of each of the $X_i$. $\forall n \geq 1$
\[Z_n := \frac{\overline{X_n} - E[\overline{X_n}]}{\sqrt{Var(\overline{X_n})}}
= \frac{\sum_{i = 0} ^ n X_i - n\mu}{\sqrt{n\sigma ^ 2}} = \frac{n\overline{X_n} - n\mu}{\sqrt{n}\sigma} = \sqrt{n}\frac{\overline{X_n} - \mu}{\sigma}\]
let $\mu$ be mean, $\sigma ^ 2$ be variance\\
for a poisson distribution, $\mu = \sigma ^ 2 = 5$
\begin{align*}
Z_n &= \sqrt{n}\frac{\overline{X_n} - \mu}{\sigma}\\
&= \sqrt{125}\frac{\overline{X_n} - 5}{\sqrt{5}}\\
&= \sqrt{5}\sqrt{25}\frac{\overline{X_n} - 5}{\sqrt{5}}\\
&= 5(\overline{X_n} - 5)\\
&Z_n \text{ converge to the probability density function of standard normal distribution } \Phi\\
P(\overline{X}_n < 5.5) &= P(5(\overline{X_n} - 5) < 5(5.5 - 5))\\
&= P(5(\overline{X_n} - 5) < 2.5)\\
&\approx \Phi(2.5)\\
&= \boxed{0.9938}
\end{align*}
\subsubsection*{Question 7 (10 points)}
Let $X_n = f(W_n, X_{n-1})$ for 
$n = 1, \ldots, P$, for some 
function $f()$. Let us define the 
value of variable $E$ as 
\begin{equation}
     E = ||C - X_P||^2,
\end{equation}
for some constant $C$. 
What is the value of the gradient 
$\frac{\partial E}{\partial X_0}$?\\
\\
\begin{align*}
X_P &= f(W_P, X_{P - 1})\\
\frac{\partial E}{\partial X_0} &= \frac{\partial||C - X_P|| ^ 2}{\partial X_0}\\
&= \frac{\partial||C - X_P|| ^ 2}{\partial X_P} \frac{\partial X_P}{\partial X_0}\\
&\text{/* dot product of a vector \textbf{v} with itself is the square of \textbf{v}'s magnitude}\\
&\text{since }\textbf{v} \cdot \textbf{v} = |\textbf{v}||\textbf{v}|\cos(0) = |\textbf{v}||\textbf{v}|(1) = |\textbf{v}| ^ 2 \textbf{ */}\\
&= \frac{\partial((C - X_P)\cdot(C - X_P))}{\partial X_P} \frac{\partial X_P}{\partial X_0}\\
&\text{/* derivative of a dot product of $\overrightarrow{a} + \overrightarrow{b}$ with respect to $a_i$:}\\
&\frac{\partial((\overrightarrow{a} + \overrightarrow{b}) \cdot (\overrightarrow{a} + \overrightarrow{b}))}{\partial b_i}
= \frac{\partial(\sum_{i = 1} ^ n a_i ^ 2 + 2a_i b_i + b_i ^ 2)}{\partial b_i}
= 2a_i + 2b_i
= 2(\overrightarrow{a} + \overrightarrow{b})_i\\
&\text{let } \overrightarrow{a} := C, \qquad \overrightarrow{b} := -X_P\\
&\text{then } \frac{\partial((C + (-X_P))\cdot(C + (-X_P)))}{\partial (-X_P)} = 2(C - X_P)(-1) = 2(X_P - C))\text{ */}\\
&= 2(X_P - C)\frac{\partial X_P}{\partial X_{P - 1}}\cdot\frac{\partial X_{P - 1}}{\partial X_{P - 2}} \,...\, \frac{\partial X_1}{\partial X_0}\\
&= \boxed{2(X_P - C)\prod_{i = 1} ^ P D_i}\\
&\text{where } D_i \text{ is the Jacobian matrix of } f(W_i, X_{i - 1})
\end{align*}

\subsection*{Question 8 (10 points)}
Let $A$ be the matrix 
$\left[ \begin{array}{ccc}
     2 & 6 & 7 \\ 
     3 & 1 & 2 \\
     5 & 3 & 4 
    \end{array} 
\right]$
and let $x$ be the column vector
$\left[ \begin{array}{c}
     2 \\ 
     3 \\
     4  
    \end{array} 
\right]$. Let $A^{T}$ and $x^{T}$ denote the transpose of $A$ and $x$ respectively. Compute $Ax$, $A^{T}$ and $x^{T}A$.\\
\\
\begin{align*}
Ax &=
\begin{bmatrix}
2 & 6 & 7\\
3 & 1 & 2\\
5 & 3 & 4\\
\end{bmatrix}
\begin{bmatrix}
2\\
3\\
4\\
\end{bmatrix}
=
\begin{bmatrix}
4 + 18 + 28\\
6 + 3 + 8\\
10 + 9 + 16\\
\end{bmatrix}
=
\begin{bmatrix}
50\\
17\\
35\\
\end{bmatrix}\\
A ^ T &=
\begin{bmatrix}
2 & 3 & 5\\
6 & 1 & 3\\
7 & 2 & 4\\
\end{bmatrix}\\
x ^ T A &=
\begin{bmatrix}
2 & 3 & 4
\end{bmatrix}
\begin{bmatrix}
2 & 6 & 7\\
3 & 1 & 2\\
5 & 3 & 4\\
\end{bmatrix}
=
\begin{bmatrix}
4 + 9 + 20 & 12 + 3 + 12 & 14 + 6 + 16
\end{bmatrix}
=
\begin{bmatrix}
33 & 27 & 36
\end{bmatrix}
\end{align*}

\subsection*{Question 9}
Find out if the following matrices are invertible. If yes, find the inverse of the matrix.\\
(a)
\begin{equation}
    \left[ 
        \begin{array}{ccc}
         6 & 2 & 3 \\ 
         3 & 1 & 1 \\
        10 & 3 & 4 
        \end{array} 
    \right]
\end{equation}
(b)
\begin{equation}
    \left[ 
        \begin{array}{ccc}
         1 & 2 & 3 \\ 
         0 & 2 & 2 \\
         1 & 4 & 5 
        \end{array} 
    \right]
\end{equation}\\
\\
a matrix is invertible if and only if its determinant is not 0\\
below calculate determinant using Laplace expansion\\
(a)
\begin{align*}
\begin{vmatrix}
6 & 2 & 3\\ 
3 & 1 & 1\\
10 & 3 & 4 
\end{vmatrix}
&=
6
\begin{vmatrix}
1 & 1\\
3 & 4\\
\end{vmatrix}
- 2
\begin{vmatrix}
3 & 1\\
10 & 4\\
\end{vmatrix}
+ 3
\begin{vmatrix}
3 & 1\\
10 & 3\\
\end{vmatrix}\\
&= 6(4 - 3) - 2(12 - 10) + 3(9 - 10) = 6 - 4 - 3 = -1
\end{align*}
determinant of (a) is 1, which is not 0, so the matrix is invertible
\begin{align*}
\left[
\begin{array}{ccc|ccc}
6 & 2 & 3 & 1 & 0 & 0\\
3 & 1 & 1 & 0 & 1 & 0\\
10 & 3 & 4 & 0 & 0 & 1
\end{array}
\right]\\
R_1 \leftarrow \frac{R_1}{6}
\left[
\begin{array}{ccc|ccc}
1 & 1 / 3 & 1 / 2 & 1 / 6 & 0 & 0\\
3 & 1 & 1 & 0 & 1 & 0\\
10 & 3 & 4 & 0 & 0 & 1
\end{array}
\right]\\
R_2 \leftarrow R_2 - 3 R_1
\left[
\begin{array}{ccc|ccc}
1 & 1 / 3 & 1 / 2 & 1 / 6 & 0 & 0\\
0 & 0 & - 1 / 2 & - 1 / 2 & 1 & 0\\
10 & 3 & 4 & 0 & 0 & 1
\end{array}
\right]\\
R_3 \leftarrow R_3 - 10 R_1
\left[
\begin{array}{ccc|ccc}
1 & 1 / 3 & 1 / 2 & 1 / 6 & 0 & 0\\
0 & 0 & - 1 / 2 & - 1 / 2 & 1 & 0\\
0 & - 1 / 3 & -1 & - 5 / 3 & 0 & 1
\end{array}
\right]\\
R_2 \leftrightarrow R_3
\left[
\begin{array}{ccc|ccc}
1 & 1 / 3 & 1 / 2 & 1 / 6 & 0 & 0\\
0 & - 1 / 3 & -1 & - 5 / 3 & 0 & 1\\
0 & 0 & - 1 / 2 & - 1 / 2 & 1 & 0
\end{array}
\right]\\
R_2 \leftarrow -3 R_2
\left[
\begin{array}{ccc|ccc}
1 & 1 / 3 & 1 / 2 & 1 / 6 & 0 & 0\\
0 & 1 & 3 & 5 & 0 & -3\\
0 & 0 & - 1 / 2 & - 1 / 2 & 1 & 0
\end{array}
\right]\\
R_3 \leftarrow -2 R_3
\left[
\begin{array}{ccc|ccc}
1 & 1 / 3 & 1 / 2 & 1 / 6 & 0 & 0\\
0 & 1 & 3 & 5 & 0 & -3\\
0 & 0 & 1 & 1 & -2 & 0
\end{array}
\right]\\
R_1 \leftarrow R_1 - \frac{R_2}{3}
\left[
\begin{array}{ccc|ccc}
1 & 0 & -1 / 2 & -3 / 2 & 0 & 1\\
0 & 1 & 3 & 5 & 0 & -3\\
0 & 0 & 1 & 1 & -2 & 0
\end{array}
\right]\\
R_2 \leftarrow R_2 - 3 R_3
\left[
\begin{array}{ccc|ccc}
1 & 0 & -1 / 2 & -3 / 2 & 0 & 1\\
0 & 1 & 0 & 2 & 6 & -3\\
0 & 0 & 1 & 1 & -2 & 0
\end{array}
\right]\\
R_1 \leftarrow R_1 + \frac{R_3}{2}
\left[
\begin{array}{ccc|ccc}
1 & 0 & 0 & -1 & -1 & 1\\
0 & 1 & 0 & 2 & 6 & -3\\
0 & 0 & 1 & 1 & -2 & 0
\end{array}
\right]
\end{align*}
inverted matrix of (a) is 
$\begin{bmatrix}
-1 & -1 & 1\\
2 & 6 & -3\\
1 & -2 & 0
\end{bmatrix}$\\
(b)
\begin{align*}
\begin{vmatrix}
1 & 2 & 3\\ 
0 & 2 & 2\\
1 & 4 & 5
\end{vmatrix}
&=
1
\begin{vmatrix}
2 & 2\\
4 & 5
\end{vmatrix}
- 2
\begin{vmatrix}
0 & 2\\
1 & 5\\
\end{vmatrix}
+ 3
\begin{vmatrix}
0 & 2\\
1 & 4
\end{vmatrix}\\
&= 1(10 - 8) - 2(0 - 2) + 3(0 - 2) = 2 + 4 - 6 = 0
\end{align*}
determinant of (b) is 0, so the matrix is not invertible
\subsubsection*{Question 10}
What is an Eigen Value of a matrix? 
What is an Eigen Vector of a matrix? 
Describe one method (any method) you 
would use to compute both of them. 
Use the above described method to compute 
the Eigen Values of the matrix:
\begin{equation}
    \left[ 
        \begin{array}{ccc}
         1 & 0 & -1 \\ 
         1 & 0 & 0 \\
        -2 & 2 & 1 
        \end{array} 
    \right]
\end{equation}
\\
let $V$ be a vector space over the field $\mathbb{C}$, $A: V \rightarrow V$ be a linear map. An element $v \in V$ is an eigenvector of $A$ if $\exists \lambda \in \mathbb{C}, Av = \lambda v$, then $\lambda$ is an eigenvalue of $A$ associated with eigenvector $v$
\begin{align*}
Av &= \lambda v\\
Av - \lambda v &= \textbf{0}\\
Av - \lambda I v &= \textbf{0}\\
(A - \lambda I)v &= \textbf{0}\\
det(A - \lambda I) &= \textbf{0}\\
A - \lambda I &=
\begin{bmatrix}
1 & 0 & -1\\ 
1 & 0 & 0\\
-2 & 2 & 1
\end{bmatrix}
-
\begin{bmatrix}
\lambda & 0 & 0\\
0 & \lambda & 0\\
0 & 0 & \lambda
\end{bmatrix}
=
\begin{bmatrix}
1 - \lambda & 0 & -1\\ 
1 & -\lambda & 0\\
-2 & 2 & 1 - \lambda
\end{bmatrix}\\
det(A - \lambda I) &=
\begin{vmatrix}
1 - \lambda & 0 & -1\\ 
1 & -\lambda & 0\\
-2 & 2 & 1 - \lambda
\end{vmatrix}\\
&= (1 - \lambda)((-\lambda)(1 - \lambda) - 0) - (1)(2 - (-\lambda)(-2))\\
&= (1 - \lambda)(-\lambda + \lambda ^ 2) - (2 - 2\lambda)\\
&= -\lambda + \lambda ^ 2 + \lambda ^ 2 - \lambda ^ 3 - 2 + 2\lambda\\
&= -\lambda ^ 3 + 2\lambda ^ 2 + \lambda - 2\\
&= -(\lambda - 1)(\lambda ^ 2 - \lambda - 2)\\
&= -(\lambda - 1)(\lambda - 2)(\lambda + 1)\\
&= 0\\
\\
\lambda_1 &= 1\\
\lambda_2 &= 2\\
\lambda_3 &= -1\\
\end{align*}
\end{document}
